\documentclass[]{article}

%opening
\title{Twitter Sentiment Analysis with Neural Networks}
\author{Shayan Sadigh \& Pedro M. Sosa}

\begin{document}

\maketitle

\section{Problem Statement}

\par We are particularly interested in understanding the concepts and functions of Neural Networks (NN). To further investigate these, we would like to build our own NN prototype to solve the very common problem of sentiment analysis. Our sentiment analysis would be focused on Twitter data, since there seems to be abundant resources for these.

\section{Background}

\par Neural networks provide a shallow imitation of how neurons in a biological neural network works. They extend upon the idea of perceptrons, which is a basic method of classifying linearly separable data. When data is not linearly separable, we can feed the output of a single perceptron to more perceptrons, chaining them together and resulting in what we call a neural network. Geometrically, we can imagine that chaining together perceptrons in this way projects our data into higher dimensions, allowing us to use a hyperplane to separate data that could not be separated in lower dimensional projections.

\par A neural network can be imagined as being made up of nodes (neurons) connected to each other by edges (synapses) that are each associated with a ``weight" (biologically, the weight would be how strongly the signal from one neuron affects the ``action potential" of the next.)

\par Neural networks are one of many machine learning algorithms used to solve complex classification problems and there is currently a lot of excitement around ``deep learning" and other extensions of neural networks in the artificial intelligence field. Although we will likely not have time to implement these more advanced techniques, we feel that it would be beneficial to get comfortable with the basics by hand-coding a neural network project.

\pagebreak
 
\section{Data Source}
Some of the data sources we are considering are:
\par - \textbf{Standford's Twitter Dataset} 
\par (https://snap.stanford.edu/data/twitter7.html)
\par - \textbf{Kaggle's Airline Sentiment}
\par (https://www.kaggle.com/crowdflower/twitter-airline-sentiment)
\par - \textbf{University of Michigan's Twitter Corpus}
\par (https://inclass.kaggle.com/c/si650winter11)
\section{Milestones}
This is a very broad project road map, which might change along the way.
\par \textbf{- Due: Oct 23} Coding a toy neural network \& researching possible NN designs (currently underway) .
\par \textbf{- Due: Oct 24} Code a simple single hidden layered NN that performs basic sentiment classification
\par \textbf{- Due: Oct 31} Experiment with different types of pre-processing twitter data for our NN input.
\par \textbf{- Due: Nov 7} Experiment with fine-tunning the NN parameters. 
\par \textbf{- Due: Nov 14} Begin comparison of our NN results with other published approaches. 
\par \textbf{- Due: Nov 21} Begin written report.

\section{Resources}
This is a list of resources gathered that might be useful for getting our project started:
\par - Dos Santos, N. C. Deep Convolutional Neural Networks for Sentiment Analysis for Short Texts.
\par - Chintala, S. Sentiment Analysis using Nerual Architectures.
\par - Yuan Y. Twitter Sentiment Analysis with Recursive Neural Networks.
\par - Tai S. K. Sentiment Analysis of Tweets: Baseline and Neural Network Models
\end{document}
